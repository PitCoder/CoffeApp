%!TEX root = ../prueba.tex

\section{Presentación}

El equipo de desarrollo está conformado por 5 personas con amplia experiencia en el campo de desarrollo de aplicaciones para aplicaciones móviles. El proyecto requiere ser visto desde dos enfoques, el primero está relacionado con la especificación del marco de trabajo Scrum, el cual indica que en un proyecto deberán existir al menos 3 roles principales:
	\begin{itemize}
		\item El product owner.
		\item El scrum master.
		\item Miembros del equipo de scrum.
	\end{itemize}

A continuación se presenta una tabla para presentar a los miembros del equipo y su rol asignado para este proyecto.

\begin{center}
\begin{longtable}{|p{0.3\textwidth}|p{0.3\textwidth}|p{0.3\textwidth}|}
	\hline
		\rowcolor{principal}
		\bf \color{white} Miembro del Equipo & \bf \color{white} Rol & \bf \color{white} Responsabilidades\\
		\hline
		 Francisco Isidoro Mera Torres & Product Owner & \begin{Titemize}
		 	\Titem Crear épicas.
			\Titem Ordenar por prioridad los elementos del Prioritized Product Backlog.
		 	\Titem Define el criterio de conclusión de los elementos del Prioritized Product Backlog.
			\Titem Crea el calendario de liberación de entregables.
			\Titem Ayuda a crear User Stories.
			\Titem Aclara dudas de las User Stories.
			\Titem Aclara requerimientos al equipo Scrum.
		 \end{Titemize}\\\hline
		 Diana Laura Mejía Mendoza & Scrum Master & \begin{Titemize}
		 	\Titem Helps identify Stakeholder(s) for the project
			\Titem Facilitates creation of Epic(s) and Personas
			\Titem Helps Product Owner in creation of the Prioritized Product Backlog and in definition of the Done Criteria
			\Titem Assists the Scrum Team in creating User Stories and their Acceptance Criteria
			\Titem Facilitates meetings of the Scrum Team to estimate User Stories
		\end{Titemize}\\\hline
		 Eliot Uriel Cedillo Vázquez & Team Member & \multirow{4}{0.3\textwidth}{\begin{Titemize}
		 	\Titem Entiende las historias de usuario del Prioritized Product Backlog. 	
		 	\Titem Se asegura del correcto entendimiento de las épicas y stakeholders.
		 	\Titem Busca aclaraciones sobre nuevos productos o cambios en productos existentes, en el Prioritized Product Backlog.
		 	\Titem Estimar las user stories aprobadas por el Product Owner
		 	\Titem Confirma las user stories  que deben realizarse en el Sprint.
		 	\Titem Desarrolla la lista de tareas con base a las user stories acordadas.
		 	\Titem Desarrolla el Sprint Backlog y el Sprint Burndown Chart.
		 	\Titem Crea entregables.
		 	\Titem Actualiza el registro de impedimentos y dependencias.
		 \end{Titemize}}\\
		 Fernando Martínez Calderón &  & \\
		 Eric Alejandro López Ayala &  & \\
		 & & \\
		 & & \\ 
		 & & \\
		 & & \\ 
		 & & \\
		 & & \\ 
		 & & \\
		 & & \\ 
		 & & \\
		 & & \\ 
		 & & \\
		 & & \\ 
		 & & \\
		 & & \\ 
		 & & \\
		 & & \\ 
		 & & \\
		 & & \\ 
		 & & \\
		 & & \\ 
		 & & \\
		 & & \\ 
		 & & \\
		\hline
\end{longtable}
\end{center}


El segundo enfoque se basa en la designación de responsabilidades en función de las siguientes áreas en las que está dividido el proyecto y la persona a cargo del área:
	\begin{itemize}[label=\color{secundario}\Coffeecup]
		\item Francisco Isidoro Mera Torres - Líder de Proyecto / DBA
		\item Diana Laura Mejia Mendoza  - Líder del Área de Análisis
		\item Eliot Uriel Cedillo Vázquez - Líder del Área de Desarrollo
		\item Fernando Martínez Calderón - Líder del Área de Diseño
		\item Eric Alejandro López Ayala - Líder del Área de Pruebas
	\end{itemize}

\section{Plan de Colaboración y Comunicación}

En esta sección se presenta una tabla que tiene como principal propósito indicar los mecanismos que serán utilizados para mentener una comunicación con los stakeholders y con los miembros del equipo más activa y detallada.


\newpage
\begin{center}
\begin{longtable}{|p{0.20\textwidth}|p{0.15\textwidth}|p{0.15\textwidth}|p{0.15\textwidth}|p{0.15\textwidth}|}
\hline
\rowcolor{principal}
	\bf\color{white}StakeHolder / Miembro del Equipo & \bf\color{white} Mensajes & \bf\color{white} Medio o Vehículo & \bf\color{white} Frecuencia & \bf\color{white} Duración \\
	\endhead
	\hline
	Scrum Team & Daily Meeting & Cara a Cara & \begin{Titemize}
		\Titem Lunes, Martes y Jueves 1:00 p.m.
		\Titem Miércoles y Viernes 2:45 p.m.
	\end{Titemize} & 15 mins\\
	\hline
	Scrum Team & Sprint Planning Meeting & Reunión cara a cara & 1 vez cada dos semanas & 2 horas\\
	\hline
	Scrum Team & Sprint Retrospective Meeting & Reunión cara a cara & 1 al finalizar cada sprint & 1 hora\\
	\hline
	\begin{Titemize}
		\Titem \getElementById[Stakeholder]{ProveedorServicio}
		\Titem \getElementById[Stakeholder]{Cocinero}
		\Titem \getElementById[Stakeholder]{Cajero}
		\Titem \getElementById[Stakeholder]{Cliente}
	\end{Titemize} & Toma de Requerimientos & \begin{Titemize}
													\Titem Reunión cara a cara.
													\Titem Llamada telefónica sujeta a grabación.
													\Titem Correo electrónico sujeta a firma.
												\end{Titemize}& 2 veces a la semana &	\begin{Titemize}
																							\Titem Mínimo: 15 mins
																							\Titem Máximo: 1 hora
																						\end{Titemize}\\
	\hline
	Profesor Ulises Vélez Saldaña & Entregas & Reunión cara a cara & Al finalizar cada sprint & Indeterminado \\
	\hline
	Product Owner & \begin{Titemize}
									\Titem Aclaración de Requerimientos.
									\Titem Aclaración de User Stories.
									\Titem Aclaración de Épicas.
								  \end{Titemize} & \begin{Titemize}
								  						\Titem Reunión cara a cara.
														\Titem Correo Electrónico.
														\Titem Mensaje de Whats App
								  				   \end{Titemize} & No aplica & No aplica.\\
	\hline
	
\end{longtable}
\end{center}