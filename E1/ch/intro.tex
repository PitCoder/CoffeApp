%!TEX root = ../prueba.tex
El presente documento tiene como propósito presentar la especificación inicial requerida para el proyecto que se llevará a cabo durante el periodo escolar 2018-2019/1 en la Escuela Superior de Cómputo para la unidad de aprendizaje \textit{Desarrollo de Aplicaciones para Dispositivos Móviles}. En este documento se presenta la justificación del por qué se lleva a cabo este proyecto, las personas que en él están involucradas y el conjunto de mecanismos, herramientas o funcionalidades que la aplicación propuesta deberá cubrir al término del proyecto y del periodo escolar. Este proyecto ha sido planeado como lo indica el marco de trabajo Scrum y con un enfoque que se basa en los principios definidos en el \href{http://agilemanifesto.org}{Manifesto para el Desarrollo de Software ágil}.

\section{Estructura del documento}

En esta sección se describen brevemente los capítulos que conforman este documento así como notas, observaciones o comentarios adicionales que tienen como propósito apoyar en su lectura. 
\begin{itemize}
	\item En el capítulo \ref{ch:bc} se describe el negocio y los aspectos más relevantes como lo son la visión, misión, organigrama y procesos de la empresa, los cuales servirán de referencia para el desarrollo y conclusión del proyecto.
	
	\item En el capítulo \ref{ch:ps} se describe mediante un diagrama de BPMN la propuesta hecha por el equipo para resolver el problema planteado en el capítulo \ref{ch:bc}. Así mismo se describe la arquitectura física y lógica que representan el esquema bajo el cual el sistema va a ser desarrollado.
	
	\item En el capítulo \ref{ch:alcance} se describen los requerimientos funcionales y no funcionales del sistema concluyendo con el diagrama de casos de uso y el mapa de navegación ideal de la aplicación móvil.
\end{itemize}

\section{Notación y Nomenclatura}

Cada capítulo utiliza la siguiente nomenclatura para identificar a los diferentes elementos que conforman al documento.

\begin{Citemize}
	\item Para identificar a los stakeholders se utiliza el prefijo \textit{ST}.
	\item Para identificar a los requerimientos funcionales se utiliza el prefijo \textit{REQMXXYY} donde $xx$ es una abreviatura para el módulo al que pertenece el requerimiento y  $yy$ es un dígito del $0$ al $99$ que sirve como identificador único para el requerimiento.
	\item Para identificar a los requerimientos no funcionales se utiliza el prefijo \textit{REQNFXX} donde $xx$ es un dígito único del $0$ al $99$.
	\item Para identificar los entregables se utiliza el siguiente esquema:EXX-CYY-SPZZZ donde XX puede ser:
			\begin{Citemize}
				\item 02 - Análisis y Diseño.
				\item 03 - Implementación.
				\item 04 - Manual de Usuario.
			\end{Citemize}
		 YY puede ser:
		 	\begin{Citemize}
				\item 01 - Módulo de Autenticación.
				\item 02 - Módulo del Proveedor del Servicio.
				\item 03 - Módulo del Cliente.
				\item 04 - Módulo de Pagos.
			\end{Citemize}
		Y ZZZ es un dígito del 0 al 999 para especificar el Sprint en el cual se está trabajando. Si el entregable no tiene las letras SPZZZ entonces se trata de la recopilación de todos los sprints.
\end{Citemize}

