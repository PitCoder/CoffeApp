%!TEX root = ../prueba.tex
En este capítulo se definen los términos del negocio que se utilizan para comprender el comportamiento del sistema.
\begin{Glosario}


	\bTerm{bCafeteria}{Cafetería}{Es una organización dedicada a ofrecer productos para satisfacer la demanda de los clientes y para lograr un buen funcionamiento pueden tener una distribución de locales.}
	\bTerm{bCliente}{Cliente}{Es una persona que interactua con el sistema de ''Coffee App" para realizar una compra en un local registrado en la aplicación móvil.}
	\bTerm{bConfirmacionDeOrden}{Confirmación de Orden}{Es cuando un cliente realiza una  orden a un local, y emite una confirmación de compra mediante el sistema, para informarle al local que el cliente acepta y así comenzar a preparar las ordenes solicitadas.}
	\bTerm{bGeolocalizacion}{Geolocalización}{Es un proceso que se encarga de determinar la posición de un dispositivo móvil.}
	\bTerm{bLocal}{Local}{Es un espacio físico dentro del Instituto Politécnico Nacional, donde se instala un comercio que ofrece todo lo relacionado a la industria alimenticia, restaurantera, cafeterías, entre otras, desde la materia prima e insumos, hasta su tranformación para la venta directa con el cliente.}
	\bTerm{bMenu}{Menú}{Es una lista donde se pueden visualizar los productos disponibles que se ofrecen en un local.}
	\bTerm{bPagoConTarjeta}{Pago con tarjeta}{Es un medio de pago emitido por una persona mediante una tarjeta financiera, se pueden efectuar compras de los productos que ofertan en un local.}
	\bTerm{bPagoEnCaja}{Pago en Caja}{Es un medio de pago emitido por una persona mediante dinero en efectivo y en un determinado lugar que se encuentre dentro de las instalaciones del local, medante este método de pago se pueden efectuar compras de productos.}
%	\bTerm{bPaquete}{Paquete}{}
	\bTerm{bPedido}{Orden}{Es una solicitud que un cliente realiza mediante el sistema "Coffee App" para la compra de varios productos o paquetes que se ofrecen en un determinado local.}
	\bTerm{bPersona}{Persona}{Es un usuario que va a interactuar con el sistema "Coffee App" mediante la aplicación \textbf{Móvil} o \textbf{WEB}.}
	\bTerm{bProducto}{Producto}{Puede ser una materia prima, un insumo, una preparación o una tranformación de un alimento para la venta con el cliente.}
	\bTerm{bPromocion}{Promoción}{Es un elemento o herramienta del marketing que tiene como objetivos específicos, informar, persuadir y recordar al público objetivo acerca de los productos que un local les ofrece.}
	\bTerm{bTiempoDeEspera}{Tiempo de Espera}{Es el tiempo transcurrido desde que la orden comienza a prepararse hasta la entrega de la orden por parte del proveedor al cual lo ha solicitdo.}
%	\bTerm{bTipoDeOrden}{Tipo de Orden}{}
%	\bTerm{bTipoDeProducto}{Tipo de Producto}{}
\end{Glosario}