%!TEX root = ../prueba.tex

\section{Reglas de negocio}

\subsection{Reglas dde Negocio Generales}

En este capítulo se detallan las reglas de negocio que rigen al módulo de \varModulo y que el sistema deberá utilizar para controlar el acceso de los usuarios a las diferentes funcionalidades del sistema.

%**************************BR001********************************
\begin{BusinessRule}{BR001}{Campos obligatorios}{}{}{}
	\BRItem[Versión] 1.0
	\BRItem[Estado] Propuesta.
	\BRItem[Propuesta por] Diana Laura Mejía Mendoza.
	\BRItem[Revisada por] Pendiente.
	\BRItem[Aprobada por] Pendiente.
	\BRItem[Descripción] Los datos proporcionados en el sistema marcados como requeridos, no se deben omitir.
	\BRItem[Sentencia] Sea un $Campo$ marcado como $Obligatorio$, $ |  campo.obligatorio = true$ 
	$ \therefore \forall campo \Rightarrow campo.valor \neq \emptyset $

	\BRItem[Ejemplo] Para iniciar sesión en el sistema \textbf{Coffee App}, el actor introducé todos los datos en los campos \textbf{Usuario} y \textbf{Contraseña} marcados como obligarios. 

\end{BusinessRule}

%**************************BR002********************************

\begin{BusinessRule}{BR002}{Información correcta}{}{}{}
	\BRItem[Versión] 1.0.
	\BRItem[Estado] Propuesta.
	\BRItem[Propuesta por] Diana Laura Mejía Mendoza.
	\BRItem[Revisada por] Pendiente.
	\BRItem[Aprobada por] Pendiente.
	\BRItem[Descripción] Todos los datos proporcionados al sistema deben pertenecer al tipo de dato establecido en el modelo de información.
	\BRItem[Sentencia] Sea $F:$ Una expresión regular que determina el formato que un campo debe cumplir con base en el modelo de información. \\ $Dato$ Los datos ingresados por el actor, en un campo del sistema.\\
	
	Si $ Dato \in F \Longrightarrow InformacionCorrecta $
	\BRItem[Ejemplos]
		\begin{itemize}
			\item El actor introduce su Nombre que contiene solo carácteres alfabéticos.
			\item El actor introduce un número de celular, el cuál solamente contiene caractéres númericos.
			\item El actor introduce un correo electrónico que contiene un símbolo '@' y cuya terminación es un dominio de correo electrónico.
		\end{itemize}

\end{BusinessRule}

%**************************BR003********************************

\begin{BusinessRule}{BR003}{Eliminación lógica de elementos}{}{}{}
	\BRItem[Versión] 1.0.
	\BRItem[Estado] Propuesta.
	\BRItem[Propuesta por] Diana Laura Mejía Mendoza.
	\BRItem[Revisada por] Pendiente.
	\BRItem[Aprobada por] Pendiente.
	\BRItem[Descripción] Un elemento sólo se puede eliminar si no tiene asociaciones con otros elementos. 

	\BRItem[Sentencia]  Si $E_x \in E_y \Longrightarrow NoSePuedeEliminarElemento$ \\
	
	Donde: $E_x  \wedge E_y$: Elementos registrados en el sistema.

	\BRItem[Ejemplos]
	\begin{itemize}
		\item Una cafetería que tiene registrados productos no se puede eliminar.
		\item Se puede eliminar la cafetería si y solo si no tiene ningún producto registrado.
	\end{itemize}

\end{BusinessRule}

%**************************BR004********************************

\begin{BusinessRule}{BR004}{Unicidad de elementos}{}{}{}
	\BRItem[Versión] 1.0.
	\BRItem[Estado] Propuesta.
	\BRItem[Propuesta por] Diana Laura Mejía Mendoza.
	\BRItem[Revisada por] Pendiente.
	\BRItem[Aprobada por] Pendiente.
	\BRItem[Descripción] Un elemento no se puede duplicar en el ámbito donde es utilizado ni registrarse más de una vez. Dada la sentencia se consideran dentro de la regla las siguientes entidades y atributos:

	\begin{itemize}
		\item Cafetería, \{nombre\}
		\item Producto, \{nombre\}
	\end{itemize}
	\BRItem[Sentencia] Sea $ E_1, E_2$: Entidades utilizadas en un mismo ámbito. \\
	$A$: Atributo de una entidad. \\
	 $ | \forall  (A \in E_1) \neq (A \in E_2)$.

	\BRItem[Ejemplo]
		\begin{itemize}
			\item Dos cafeterías con diferentes nombres.
			\item Cafetería El Kiosquito.
			\item Cafetería Zen Garden
		\end{itemize}

\end{BusinessRule}

%Negocio


%**********************************************************
\subsection{Reglas de Negocio}
%**********************************************************


%**************************BR-MAU001********************************

\begin{BusinessRule}{BR-MAU001}{Número máximo de intentos}{}{}{}
	\BRItem[Versión] 1.0.
	\BRItem[Estado] Propuesta.
	\BRItem[Propuesta por] Diana Laura Mejía Mendoza.
	\BRItem[Revisada por] Pendiente.
	\BRItem[Aprobada por] Pendiente.
	\BRItem[Descripción]El número máximo de intentos fallidos para ingresar al sistema será de 3, de lo contrario la cuenta se bloqueará hasta que el administrador cambie el estado de la cuenta.
	
	\BRItem[Sentencia] Sea $Num_I$: Número de intentos que el usuario realiza para ingresar al sistema. \\

	Si $Num_I = 3 \Longrightarrow CuentaBloqueada$.


\end{BusinessRule}

%**************************BR-MAU002********************************

\begin{BusinessRule}{BR-MAU002}{Sesión activa}{}{}{}


\end{BusinessRule}

%**************************BR-MAU003********************************

\begin{BusinessRule}{BR-MAU003}{Contraseña válida}{}{}{}
	\BRItem[Versión] 1.0.
	\BRItem[Estado] Propuesta.
	\BRItem[Propuesta por] Eric Alejandro López Ayala.
	\BRItem[Revisada por] Pendiente.
	\BRItem[Aprobada por] Pendiente.
	\BRItem[Descripción] Una constraseña únicamente sera válida si tiene una longitud mínima de 8 carácteres, contiene al menos un carácter mayúscula, un carácter minúscula, un dígito y alguno de los siguientes carácteres especiales: '.' ',' '-' '\_'
	\BRItem[Sentencia] Sea $F:$ Una expresión regular que determina el formato que un campo debe cumplir con base en el modelo de información. \\ $Dato$ Los datos ingresados por el actor, en un campo del sistema.\\
	
	Si $ Dato \in F \Longrightarrow ContrasenaValida $
	\BRItem[Ejemplos]
	\begin{itemize}
		\item El actor introduce una contraseña de al menos 8 caráteres de longitud e incluye al menos un carácter minúscula, un carácter mayúscula, un dígito y alguno de los siguientes carácteres especiales: '.' ',' '-' '\_'
	\end{itemize}
	
\end{BusinessRule}

%**************************BR-MC005********************************

\begin{BusinessRule}{BR-MC005}{Calificación de un producto}{}{}{}
	\BRItem[Versión] 1.0.
	\BRItem[Estado] Propuesta.
	\BRItem[Propuesta por] Diana Laura Mejía Mendoza.
	\BRItem[Revisada por] Pendiente.
	\BRItem[Aprobada por] Pendiente.
	\BRItem[Descripción] Cada producto ofertado en un local tendrá una calificación, sólo un \textbf{Cliente} o una \textbf{Persona} que realizó la compra y se le entrego el producto podrá calificarlo. Así mismo, se pueden realizar $n$ calificaciones por persona del mismo producto, esto será con base en las veces que compró y se le entregó el producto.
	\BRItem[Sentencia] 
	
\end{BusinessRule}

%**************************BR-MC007********************************

\begin{BusinessRule}{BR-MC007}{Notificaciones}{}{}{}
	\BRItem[Versión] 1.0.
	\BRItem[Estado] Propuesta.
	\BRItem[Propuesta por] Diana Laura Mejía Mendoza.
	\BRItem[Revisada por] Pendiente.
	\BRItem[Aprobada por] Pendiente.
	\BRItem[Descripción] Cada producto ofertado en un local tendrá una calificación, sólo un \textbf{Cliente} o una \textbf{Persona} que realizó la compra y se le entrego el producto podrá calificarlo. Así mismo, se pueden realizar $n$ calificaciones por persona del mismo producto, esto será con base en las veces que compró y se le entregó el producto.
	\BRItem[Sentencia] 
	
\end{BusinessRule}

%**************************BR-MC008********************************

\begin{BusinessRule}{BR-MC008}{Clasificación de notificaciones}{}{}{}
	\BRItem[Versión] 1.0.
	\BRItem[Estado] Propuesta.
	\BRItem[Propuesta por] Diana Laura Mejía Mendoza.
	\BRItem[Revisada por] Pendiente.
	\BRItem[Aprobada por] Pendiente.
	\BRItem[Descripción] Cada producto ofertado en un local tendrá una calificación, sólo un \textbf{Cliente} o una \textbf{Persona} que realizó la compra y se le entrego el producto podrá calificarlo. Así mismo, se pueden realizar $n$ calificaciones por persona del mismo producto, esto será con base en las veces que compró y se le entregó el producto.
	\BRItem[Sentencia] 
	
\end{BusinessRule}