%!TEX root = ../../prueba.tex

\begin{UseCase}{CUMC14M}{Calificar Cafetería}{}
	\UCitem{Versión}{0.1}
	\UCitem{Elaboró}{Eliot Uriel Cedillo Vázquez}
	\UCitem{Supervisó}{Diana Laura Mejía Mendoza}
	\UCitem{Prioridad}{Baja}
	\UCitem{Complejidad}{Baja}
	\UCitem{Volatilidad}{Baja}
	\UCitem{Madurez}{}
	\UCsection{Atributos}
	\UCitem{Actor}{Cliente}
	\UCitem{Propósito}{Valorar la satisfacción de atención y servicio de un local de alguna cafetería.}
	\UCitem{Entradas}{\begin{Titemize}
		\Titem \getElementById[Entidad]{local.nuRating}.
	\end{Titemize }
	\UCitem{Salidas}{Calificación final}
	\UCitem{Precondiciones}{
		\begin{Titemize}
			\Titem Se debe haber pedido almenos un pedido en la cafeteria.
			\Titem Almenos un pedido del cliente que califica debe haber sido entregado.
                    \end{Titemize}}	
	\UCitem{Postcondiciones}{Se agregara la calificación a la calificación general de la cafetería}
	\UCitem{Reglas de Negocio}{}
	\UCitem{Errores}{
		\begin{Titemize}
			\Titem No se pudo registrar la calificación.
			\Titem No se pude califcar la cafeteria.
		\end{Titemize}
		
	}
	\UCitem{Viene de}{}
	\end{UseCase}
	
	
	\begin{UCtrayectoria}
		\UCpaso[\UCactor]Buscara un local en la interfaz de usuario 13.\refTray{A}
		\UCpaso[\UCactor]Presionara sobre el local que deseé.
		\UCpaso Muestra la interfaz de usuario 15.
		\UCpaso[\UCactor] Presiona sobre la calificación del local.
		\UCpaso Comprueba si el cliente ah realizado un pedido en el local.\refTray{B}
		\UCpaso Comprueba si almenos un pedido realizado por el cliente ah sido entregado.\refTray{C}
		\UCpaso Muestra la interfaz de usuario ??.%Aquí no se si se mostrara otra interfaz de usuario o simplemente sera un pop-up para poner la calificación con las estrellitas
		\UCpaso[\UCactor]Presiona el numero de estrellas que otorgara para calificar el local.
		\UCpaso[\UCactor] Dara click en el boton \IUbutton{Aceptar}\refTray{D}
		\UCpaso Guarda la calificación del local.
		\UCpaso Promediara la calificación nueva con la calificación general del local.
		\UCpaso Muestra el mensaje:"Calificación guardada con exito gracias por tu opinion :)".
%El mensaje puede cambiar segun lo que sea mas conveniente
		\UCpaso Muestra la interfaz de usuario 15.
	\end{UCtrayectoria}
	

	\begin{UCtrayectoriaA}{A}{Entrega de un pedido activo realizado}
		\UCpaso Comprueba el estado de un pedido.
		\UCpaso Envía una notificación si el estado del pedido es igual a entregado.
		\UCpaso[\UCactor]Presiona la notificación.
		\UCpaso Muestra la Interfaz de usuario 15.%¿la interfaz de usuario sera generica nadamas cambiando el nombre del local o existira una interfaz de usuario diferentepara cada franquicia de cafeteria?
		\UCpaso[\UCactor] Presiona sobre la calificación del local.
		\UCpaso Muestra la interfaz de usuario ??.%Aquí no se si se mostrara otra interfaz de usuario o simplemente sera un pop-up para poner la calificación con las estrellitas
		\UCpaso[\UCactor]Presiona el numero de estrellas que otorgara para calificar el local.
		\UCpaso[\UCactor] Dara click en el boton \IUbutton{Aceptar}\refTray{D}
		\UCpaso Guarda la calificación del local.
		\UCpaso Promediara la calificación nueva con la calificación general del local.
		\UCpaso Muestra el mensaje:"Calificación guardada con exito gracias por tu opinion :)".%El mensaje puede cambiar segun lo que sea mas conveniente
		\UCpaso Muestra la interfaz de usuario 15.
	\end{UCtrayectoriaA}

	\begin{UCtrayectoriaA}{B}{Nose han realizado pedidos en ese local}
		\UCpaso Muestra el mensaje:"No se puede calificar: lo sentimos aun no haz realizado pedidos en este local :("
		\UCpaso Muestrala interfaz de usuario 13.
	\end{UCtrayectoriaA}

	\begin{UCtrayectoriaA}{C}{ningun pedido realizado en el local ah sido entregado}
		\UCpaso Muestra el mensaje: "No se puede calificar: lo sentimos ninguno de tus pedidos ah sido entregado satisfactoriamente :("
		\UCpaso Muestrala interfaz de usuario 13.
	\end{UCtrayectoriaA}

	\begin{UCtrayectoriaA}{D}{Cancelación de calificación}
		\UCpaso[\UCactor]Dara click en el boton \IUbutton{Cancelar}
		\UCpaso No guardara la calificación.
		\UCpaso Muestra la interfaz de usuario 15
	\end{UCtrayectoriaA}