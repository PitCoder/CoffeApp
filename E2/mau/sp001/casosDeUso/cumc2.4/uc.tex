%!TEX root = ../../prueba.tex

\begin{UseCase}{CUMC2.4}{Configurar Cuenta}{
Cuando una \getElementById[Stakeholder]{Persona} quiere configurar su cuenta, deberá dar clic en el botón \textbf{Configurar Cuenta} de la pantalla %Referencia al mockup de la pantalla: 
\textbf{Cuenta}, se mostrará la pantalla %Referencia al mockup de la pantalla: 
\textbf{Editar Cuenta} en la cual podrá editar tanto sus datos personales como lo son: \textbf{nombre} y \textbf{apellidos}, además podrá editar la \textbf{contraseña} de su cuenta de usuario, esto provee de un mecanismo al usuario para poder mantener actualizada su información y modificarla cuando lo desee.\\}

	%DESCRIPCIÓN DEL CASO DE USO
	\UCitem{Versión}{0.1}
	\UCitem{Elaboró}{López Ayala Eric Alejandro}
	\UCitem{Supervisó}{Diana Laura Mejía Mendoza}
	\UCitem{Prioridad}{Baja}
	\UCitem{Complejidad}{Baja}
	\UCitem{Volatilidad}{Baja}
	\UCitem{Madurez}{Media}
	\UCsection{Atributos}
	\UCitem{Actor}{\getElementById[Stakeholder]{Persona}}
	\UCitem{Propósito}{Definir un mecanismo mediante el cual una \getElementById[Stakeholder]{Persona} pueda modificar y mantener actualizada la información de su cuenta cuando así lo desee.}
	\UCitem{Entradas}{
		\begin{Titemize}
			%DATOS PERSONALES DE LA PERSONA
			\Titem \getElementById[Entidad]{Persona.nombre}.
			\Titem \getElementById[Entidad]{Persona.primerApellido}.
			\Titem \getElementById[Entidad]{Persona.segundoApellido}. 

			%DATOS DE CUENTA
			\Titem \getElementById[Entidad]{Cuenta.contrasena}.
		\end{Titemize}

	}
	\UCitem{Salidas}{
		\begin{Titemize}
			%MSG01: Operación realizada exitosamente - Mensaje genérico
			%En caso de requerir un mensaje más personalizado agregar el mensaje...
			\Titem \getElementById[MSG]{MSG01}
		\end{Titemize}
	}
	\UCitem{Precondiciones}{
		\begin{Titemize}
			\Titem La persona debe de tener una cuenta de usuario registrada en el sistema.
		\end{Titemize}
	}	
	\UCitem{Postcondiciones}{
		\begin{Titemize}
			\Titem La persona tendrá sus datos personales y sus datos de cuenta actualizados.
		\end{Titemize}
	}
	\UCitem{Reglas de Negocio}{
		\begin{Titemize}
			\Titem
			%Regla de negocio: Contraseña válida
			%\Titem \getElementById[BR]{BR-MAU003}
		\end{Titemize}
	}
	%Aqui me siento que faltan mas mensajes...
%	\UCitem{Errores}{
%		\begin{Titemize}
%			\Iitem \UCerr{Uno}{Los campos marcados en el formulario como obligatorios están vacios}{se muestra el mensaje \getElementById[MSG]{MSG2}}
%			
%			\Titem \UCerr{Dos}{Alguno de los campos dentro del formulario no cumple con el formato establecido dentro de las reglas de negocio correspondiente a dicho campo}{se muestra el mensaje \getElementById[MSG]{MSG6}}
%		\end{Titemize}
%	}
	\UCitem{Viene de}{No aplica.}
\end{UseCase}
		
	\begin{UCtrayectoria}
		\UCpaso[\UCactor] Ingresa en el navegador de su preferencia la ruta \textit{''http://coffeeapp.com/''}.
		\UCpaso Verifica que no haya una sesión activa para el actor con base en la regla de negocio \getElementById[BR]{BR-MAU002}.\refTray{A}
		\UCpaso Solicita al actor ingresar los datos de entrada como se muestra en la pantalla \getElementById[IU]{IUMAU01W}.
		\UCpaso[\UCactor] Ingresa los datos de entrada en la pantalla \getElementById[IU]{IUMAU01W}.
		\UCpaso[\UCactor] Indica que ha concluido de ingresar los datos de entrada en la pantalla \getElementById[IU]{IUMAU01W} presionando el botón \IUbutton{Iniciar Sesión}.
		\UCpaso Verifica que los campos marcados como obligatorios no estén vacíos con base en la regla de negocio \getElementById[BR]{BR001}.\refTray{B}
		\UCpaso Verifica que exista una cuenta asociada al actor con el \getElementById[Entidad]{Cuenta.nombreUsuario} y obtiene la información de la cuenta.\refTray{C}
		\UCpaso Verifica que la contraseña ingresada con por el actor corresponda con la \getElementById[Entidad]{Cuenta.contrasena} de la cuenta.\refTray{D}
		\UCpaso Verifica que la cuenta del actor no tenga asociado el estado \textbf{Bloqueado} con base en la máquina de estados \getElementById[MAQ]{MAQ-MAU01}.\refTray{E}
		\UCpaso Redirecciona al actor a su página de bienvenida que le corresponda a su rol.	\end{UCtrayectoria}
	

	\begin{UCtrayectoriaA}{A}{Ya existe una sesión activa y asociada con el actor.}
		\UCpaso fads
	
	\end{UCtrayectoriaA}
	