%!TEX root = ../../prueba.tex

\begin{UseCase}{CUMP4.3}{Cambiar Forma de Pago}{Cuando el \getElementById[Stakeholder]{Cliente} realize un pedido a un local y quiere pagar ya sea via Pay Pal, Tarjeta de Credito o en efectivo, el sistema le dara la opcion de poder cambiar su forma de pago a la manera que el prefiera.}
	\UCitem{Versión}{0.1}
	\UCitem{Elaboró}{Fernando Alonso Martínez Calderón}
	\UCitem{Supervisó}{Diana Laura Mejía Mendoza}
	\UCitem{Prioridad}{Alta}
	\UCitem{Complejidad}{Media}
	\UCitem{Volatilidad}{Baja}
	\UCitem{Madurez}{Media}
	\UCsection{Atributos}
	\UCitem{Actor}{\getElementById[Stakeholder]{Cliente}}
	\UCitem{Propósito}{.}
	\UCitem{Entradas}{\begin{Titemize}
		\Titem \getElementById[Entidad]{FormaPago.tipo}.
	\end{Titemize}}
	\UCitem{Salidas}{\begin{Titemize}
		\Titem \getElementById[Entidad]{FormaPAgo
	\end{Titemize}}
	\UCitem{Precondiciones}{\begin{Titemize}
		\Titem El cliente debe haber iniciado sesión en la aplicación.
	\end{Titemize}}	
	\UCitem{Postcondiciones}{\begin{Titemize}
		\Titem El cliente podrá cambiar su forma de pago.
	\end{Titemize}}
	\UCitem{Reglas de Negocio}{No aplica}
	\UCitem{Errores}{\begin{Titemize}
		
	\end{Titemize}}
	\UCitem{Viene de}{\getElementById[CU]{CUMAU1}}
	\end{UseCase}
	
	\begin{UCtrayectoria}
		\UCpaso[\UCactor] Indica que va a realizar un cambio en su forma de pago presionando el icono \moneyIcon{money} como se muestra en la pantalla \getElementById[IU]{UIMC2}.
		\UCpaso
	\end{UCtrayectoria}