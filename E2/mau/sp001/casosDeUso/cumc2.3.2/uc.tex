%!TEX root = ../../prueba.tex

\begin{UseCase}{CUMC07}{Consultar producto}{
Permite visualizar la información general de un producto para que una \getElementById[Stakeholder]{Persona} o un \getElementById[Stakeholder]{Cliente} se percate si el producto de interes, es el que el requiere. \\  
Es importante saber si el producto está diponible o se ha agotado ya que solo podrá visualizar la descripción general del producto si se encuentra disponible. \\ Un producto pasa al estado \textbf{Agotado}, debido a que en el local se terminó la existencia de dicho producto, o el local donde ofertan el producto ha finalizado su jornada laboral. \\
Por otra parte al consultar la información del producto se podrá visualizar el promedio de la calificación que le han otorgado, es importante ya que ayuda a incrementar la referencia que tiene este producto y así con base en la calificación y el criterio del cliente realice su compra.\\
Una gran 
}
	\UCitem{Versión}{0.1}
	\UCitem{Elaboró}{Diana Laura Mejía Mendoza}
	\UCitem{Supervisó}{Francisco Isidoro Mera Torres}
	\UCitem{Prioridad}{Alta}
	\UCitem{Complejidad}{Media}
	\UCitem{Volatilidad}{Baja}
	\UCitem{Madurez}{Media}
	\UCsection{Atributos}
	\UCitem{Actor}{\begin{Titemize}
	\Titem \getElementById[Stakeholder]{Persona}
	\Titem \getElementById[Stakeholder]{Cliente}
	\end{Titemize}}
	\UCitem{Propósito}{Proporcionar un mecanismo .}
	\UCitem{Entradas}{\begin{Titemize}
		\Titem \getElementById[Entidad]{Cuenta.nombreDeUsuario}.
		\Titem \getElementById[Entidad]{Cuenta.contrasena}.
		\Titem \getElementById[Entidad]{Acceso.numeroIntentos}.
	\end{Titemize}}
	\UCitem{Salidas}{\begin{Titemize}
		\Titem \getElementById[MSG]{MSG03}
	\end{Titemize}}
	\UCitem{Precondiciones}{\begin{Titemize}
		\Titem La persona que va a ingresar al sistema debe estar registrada en el sistema.
		\Titem No debe existir una sesión activa.
	\end{Titemize}}	
	\UCitem{Postcondiciones}{\begin{Titemize}
		\Titem La persona tiene acceso a la página de bienvenida que le corresponde a su 
		\Titem La persona tiene
	\end{Titemize}}
	\UCitem{Reglas de Negocio}{
		\begin{Titemize}
			\Titem \getElementById[BR]{BR-MAU01}
		\end{Titemize}
	}
	\UCitem{Errores}{
		\begin{Titemize}
			\Titem \UCerr{Uno}{Los campos marcados en el formulario como obligatorios están vacíos}{se muestra el mensaje \getElementById[MSG]{MSG02}}
		\end{Titemize}
	}
	\UCitem{Viene de}{No aplica.}
	\end{UseCase}
	
	
	\begin{UCtrayectoria}
		\UCpaso[\UCactor] Solicita consultar la información de un producto dando clic en el producto deseado de la pantalla \getElementById[UIMC07]{Carrito de compras}.
		\UCpaso Verifica que el estado del producto sea \textbf{Disponible} con base en el \textbf{Modelo de ciclo de vida de un producto}.\refErr{Uno}
		\UCpaso Obtiene la foto, el nombre, la descripción y precio del producto seleccionado.
		\UCpaso Obtiene el promedio de la calificación que los clientes han asignado al producto con base en la regla de negocio \textbf{REGLA DE NEGOCIO}.
		\UCpaso Habilita el campo \textbf{Ingresar nota}.
		\UCpaso Muestra la pantalla \getElementById[UIMC07]{Consultar producto} con la información obtenida.
		\UCpaso [\UCactor] Ingresa la información solicitada.
		\UCpaso [\UCactor] \label{ConsultarProducto-Productos} Selecciona el número de productos que requiere comprar.
		\UCpaso Calcula el total de dinero con base en los productos solicitados en el paso \ref{ConsultarProducto-Productos}
		\UCpaso [\UCactor] Solicita agregar el producto a su carrito de compras presionando el botón \IUbutton{Agregar al carrito}.\refTray{B}
		\UCpaso Verifica que exista la cantidad de productos solicitados con base en la regla de negocio \textbf{REGLA DE NEGOCIO}.\refTray{C}
		\UCpaso Muestra la pantalla \getElementById[UIMC07]{Carrito de compras}.
	\end{UCtrayectoria}
	

	\begin{UCtrayectoriaA}{A}{Cuando el estado del producto es \textbf{Agotado}.}
		\UCpaso Muestra el mensaje \textbf{AGOTADO} en el producto solitado de la pantalla .
	
	\end{UCtrayectoriaA}
	