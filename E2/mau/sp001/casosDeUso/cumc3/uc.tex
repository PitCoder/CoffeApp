%!TEX root = ../../prueba.tex

\begin{UseCase}{CUMC3}{Consultar Locales}{Cuando el \getElementById[Stakeholder]{Cliente} quiere realizar un pedido a un local, la aplicación determina la ubicación del cliente en la Tierra mediante el módulo de GPS que proporciona los valores de longitud y latitud que van a ser utilizados para encontrar los locales más cercanos a la posición del cliente. Así mismo, si el cliente conociese el nombre de un local en particular podrá realizar su búsqueda en el sistema.}
	\UCitem{Versión}{0.1}
	\UCitem{Elaboró}{Francisco Isidoro Mera Torres}
	\UCitem{Supervisó}{Diana Laura Mejía Mendoza}
	\UCitem{Prioridad}{Alta}
	\UCitem{Complejidad}{Media}
	\UCitem{Volatilidad}{Baja}
	\UCitem{Madurez}{Media}
	\UCsection{Atributos}
	\UCitem{Actor}{\getElementById[Stakeholder]{Cliente}}
	\UCitem{Propósito}{Definir un mecanismo mediante el cual el Cliente pueda encontrar locales con base en su posición o experiencia y así seleccionar uno para agregar o quitar los artículos que va a consumir así como para conocer más información acerca del local.}
	\UCitem{Entradas}{\begin{Titemize}
		\Titem \getElementById[Entidad]{local.nombreLocal}.
		\Titem \getElementById[Entidad]{local.nuLatitud}.
		\Titem \getElementById[Entidad]{local.nuLatitud}.
	\end{Titemize}}
	\UCitem{Salidas}{\begin{Titemize}
		\Titem \getElementById[Entidad]{local.nombreLocal}.
		\Titem \getElementById[Entidad]{local.horaInicio}.
		\Titem \getElementById[Entidad]{local.horaFin}.
		\Titem \getElementById[Entidad]{local.nuRating}.
		\Titem \getElementById[Entidad]{local.foto}.
	\end{Titemize}}
	\UCitem{Precondiciones}{\begin{Titemize}
		\Titem El cliente debe haber iniciado sesión en la aplicación.
		\Titem El cliente debe tener activado y habilitado el servicio de geolocalización en su dispositivo móvil.	
		\Titem Debe existir al menos un local en estado de \textbf{Publicado} para ser desplegado.		
	\end{Titemize}}	
	\UCitem{Postcondiciones}{\begin{Titemize}
		\Titem El cliente podrá consultar la lista de productos de los locales que se despliegan en pantalla.
		\Titem El cliente podrá consultar la información del local como lo es su ubicación
	\end{Titemize}}
	\UCitem{Reglas de Negocio}{No aplica}
	\UCitem{Errores}{\begin{Titemize}
		\Titem \UCerr{Uno}{Cuando el cliente no tenga habilitado el servicio de geolocalización en su dispositivo}{se desplegará en la pantalla el mensaje \getElementById[MSG]{MSGMOR1}.}
		\Titem \UCerr{Dos}{Cuando no existe al menos un local cerca de la posición del cliente}{se desplegará en la pantalla el mensaje \getElementById[MSG]{MSGMOR2}.}
		\Titem \UCerr{Tres}{Cuando no exista al menos un local con un nombre que contenga las palabras ingresadas por el Cliente}{se desplegará en la pantalla el mensaje \getElementById[MSG]{MSG3}.}
	\end{Titemize}}
	\UCitem{Viene de}{\getElementById[CU]{CUMAU1}}
	\end{UseCase}
	
	\begin{UCtrayectoria}
		\UCpaso[\UCactor] Solicita consultar los locales más cercanos a su hubicación, presionando el icono \homeIcon{Home} del menú \getElementById[IU]{MN01}. \refTray{A}
		\UCpaso Verifica que la opción de geolocalización del dispositivo del actor se encuentre habilitada.\refTray{B}
		\UCpaso \label{CUMC3:Return} Obtiene la longitud y la latitud de la ubicación actual del dispositivo.\refTray{C}
		\UCpaso Obtiene la información de al menos un local que se encuentre a una distancia de 2km a la redonda del actor.\refErr{Dos}
		\UCpaso \label{CUMC3:Fin}Muestra la información obtenida de los locales como se muestra en la pantalla \getElementById[IU]{IUMC3}.
	\end{UCtrayectoria}
	
		\begin{UCtrayectoriaA}{A}{El actor consulta los locales iniciando sesión en al sistema con la aplicación móvil}
		\UCpaso Solicita consultar los locales más cercanos a su hubicación 
		\UCpaso[\UCactor] Habilita la opción para la geolocalización presionando el botón \IUbutton{Permitir}.
		\UCpaso Regresa al paso \ref{CUMC3:Return} de la trayectoria principal.
	\end{UCtrayectoriaA}
	
	
	\begin{UCtrayectoriaA}{A}{La opción de geolocalización del dispositivo no está activada}
		\UCpaso Solicita al actor habilitar la opción de geolocalización como se muestra en la pantalla \getElementById[IU]{IUMC3}
		\UCpaso[\UCactor] Habilita la opción para la geolocalización presionando el botón \IUbutton{Permitir}.
		\UCpaso Regresa al paso \ref{CUMC3:Return} de la trayectoria principal.
	\end{UCtrayectoriaA}
	
	\begin{UCtrayectoriaA}{B}{El actor realizará la búsqueda de un local por nombre.}
		\UCpaso[\UCactor] Indica que va a realizar la búsqueda por nombre de un local ingresando texto como se muestra en la pantalla \getElementById[IU]{UIMC3}.
		\UCpaso Obtiene el texto ingresado por el actor.
		\UCpaso Obtiene la información de los locales cuyo nombre contenga las palabras ingresadas por el actor.\refErr{Tres}
		\UCpaso Muestra la información obtenida de los locales como se muestra en la pantalla \getElementById[IU]{UIMC3}.
	\end{UCtrayectoriaA}

	
	\UCExtensionPoint{Consultar Información de Local}{El actor quiere conocer la información del local.}{A partir del paso \ref{CUMC3:Fin}}{\getElementById[CU]{CUMC3.1}}
	
	\UCExtensionPoint{Consultar Lista de Productos}{El actor va a realizar la consulta de la lista de productos disponibles de un local seleccionado.}{A partir del paso \ref{CUMC3:Fin}}{\getElementById[CU]{CUMC3.2}}