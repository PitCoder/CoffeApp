%!TEX root = ../../prueba.tex

\begin{UseCase}{CUMC2.3}{Consultar Historial de Pedidos}{
Cuando un \getElementById[Stakeholder]{Cliente} quiere consultar su historial de pedidos entregados, deberá dar clic en el botón \textbf{Pedidos} de la pantalla %Referencia al mockup de la pantalla:
\textbf{Cuenta}, se mostrará la pantalla %Referencia al mockup de la pantalla:
\textbf{Historial de Pedidos} en la cual se mostrarán todos los pedidos que han sido entregados al cliente, se podrá visualizar la información de los pedidos como lo es: \textbf{Fecha de Entrega} y \textbf{Local} a donde se realizó el pedido. Se provee de un punto de acceso para poder consultar detalles de los \textbf{Pedidos Entregados} y consultar los \textbf{Pedidos Actuales}.}

	%DESCRIPCIÓN DEL CASO DE USO
		\UCitem{Versión}{0.1}
		\UCitem{Elaboró}{López Ayala Eric Alejandro}
		\UCitem{Supervisó}{Diana Laura Mejía Mendoza}
		\UCitem{Prioridad}{Media}
		\UCitem{Complejidad}{Baja}
		\UCitem{Volatilidad}{Baja}
		\UCitem{Madurez}{Media}
		\UCsection{Atributos}
		\UCitem{Actor}{\getElementById[Stakeholder]{Persona}}
		\UCitem{Propósito}{Definir un mecanismo mediante el cual un \getElementById[Stakeholder]{Cliente} pueda consultar el historial de sus pedidos, pudiendo consultar tanto sus pedidos entregados y sus pedidos activos mediante un punto de acceso.}
		\UCitem{Entradas}{
			\begin{Titemize}
				\Titem Ninguna
			\end{Titemize}
		}
		\UCitem{Salidas}{
			\begin{Titemize}
				\Titem Ninguna
			\end{Titemize}
		}
		\UCitem{Precondiciones}{
			\begin{Titemize}
				\Titem Debe de haber por lo menos un pedido registrado.
			\end{Titemize}
		}	
		\UCitem{Postcondiciones}{
			\begin{Titemize}
				\Titem Ninguno.
			\end{Titemize}
		}
		\UCitem{Reglas de Negocio}{
			\begin{Titemize}
				\Titem
			\end{Titemize}
		}
	%Aqui me siento que faltan mas mensajes...
%	\UCitem{Errores}{
%		\begin{Titemize}
%			\Iitem \UCerr{Uno}{Los campos marcados en el formulario como obligatorios están vacios}{se muestra el mensaje \getElementById[MSG]{MSG2}}
%			
%			\Titem \UCerr{Dos}{Alguno de los campos dentro del formulario no cumple con el formato establecido dentro de las reglas de negocio correspondiente a dicho campo}{se muestra el mensaje \getElementById[MSG]{MSG6}}
%		\end{Titemize}
%	}
		\UCitem{Viene de}{No aplica.}
\end{UseCase}
		
	\begin{UCtrayectoria}
		\UCpaso[\UCactor] Ingresa en el navegador de su preferencia la ruta \textit{''http://coffeeapp.com/''}.
		\UCpaso Verifica que no haya una sesión activa para el actor con base en la regla de negocio \getElementById[BR]{BR-MAU002}.\refTray{A}
		\UCpaso Solicita al actor ingresar los datos de entrada como se muestra en la pantalla \getElementById[IU]{IUMAU01W}.
		\UCpaso[\UCactor] Ingresa los datos de entrada en la pantalla \getElementById[IU]{IUMAU01W}.
		\UCpaso[\UCactor] Indica que ha concluido de ingresar los datos de entrada en la pantalla \getElementById[IU]{IUMAU01W} presionando el botón \IUbutton{Iniciar Sesión}.
		\UCpaso Verifica que los campos marcados como obligatorios no estén vacíos con base en la regla de negocio \getElementById[BR]{BR001}.\refTray{B}
		\UCpaso Verifica que exista una cuenta asociada al actor con el \getElementById[Entidad]{Cuenta.nombreUsuario} y obtiene la información de la cuenta.\refTray{C}
		\UCpaso Verifica que la contraseña ingresada con por el actor corresponda con la \getElementById[Entidad]{Cuenta.contrasena} de la cuenta.\refTray{D}
		\UCpaso Verifica que la cuenta del actor no tenga asociado el estado \textbf{Bloqueado} con base en la máquina de estados \getElementById[MAQ]{MAQ-MAU01}.\refTray{E}
		\UCpaso Redirecciona al actor a su página de bienvenida que le corresponda a su rol.	\end{UCtrayectoria}
	

	\begin{UCtrayectoriaA}{A}{Ya existe una sesión activa y asociada con el actor.}
		\UCpaso fads
	
	\end{UCtrayectoriaA}