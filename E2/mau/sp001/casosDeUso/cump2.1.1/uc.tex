%!TEX root = ../../prueba.tex
\begin{UseCase}{CUMP2.1.1}{Agregar Nueva Forma de Pago}{Cuando el \getElementById[Stakeholder]{Cliente} requiera que sus ordenes sean pagadas con algún otro metodo de pago distinto a efectivo como Paypal o Tarjeta de Crédito podrá guardar la información correspondiente para cada metodo de pago que sera usado en el pago de las futuras ordenes segun el \getElementById[Stakeholder]{Cliente} requiera.}
	\UCitem{Versión}{0.1}
	\UCitem{Elaboró}{Eliot Uriel Cedillo Vázquez}
	\UCitem{Supervisó}{Diana Laura Mejía Mendoza}
	\UCitem{Prioridad}{Alta}
	\UCitem{Complejidad}{Media}
	\UCitem{Volatilidad}{Media}
	\UCitem{Madurez}{Media}
	\UCsection{Atributos}
	\UCitem{Actor}{\getElementById[Stakeholder]{Cliente}}
	\UCitem{Propósito}{Tener distintas opciones de pago con las que se realizaran las ordenes en los locales}
	\UCitem{Entradas}{
		\begin{Titemize}
			\Titem \getElementById[Entidad]{paypal.correoElectronico}.
			\Titem \getElementById[Entidad]{paypal.contraseña}.
			\Titem \getElementById[Entidad]{tarjeta.numeroTarjeta}.
			\Titem \getElementById[Entidad]{tarjeta.fechaVencimiento}.
			\Titem \getElementById[Entidad]{tarjeta.claveSeguridad}.
		\end{Titemize}
	}
	\UCitem{Salidas}{Formas de pago registradas}
	\UCitem{Precondiciones}{
		\begin{Titemize}
			\Titem No tener agregada la información correspondiente  a la forma de pago que se quiere agregar.
                    \end{Titemize}}	
	\UCitem{Postcondiciones}{Nueva forma de pago agregada}
	\UCitem{Reglas de Negocio}{
		\begin{Titemize}
			\Titem \getElementById[BR]{BR001}
			\Titem \getElementById[BR]{BR002}
			\Titem \getElementById[BR]{BR004}
		\end{Titemize}
	}
	\UCitem{Errores}{
		\begin{Titemize}
			\Titem \UCerr{Uno}{EL cliente introduce un campo con un formato incorrecto}{Se muestra el mensaje \getElementById[MSG]{MSG6}.}
			\Titem\UCerr{Dos}{El cliente no introduce un campo obligatorio}{Se muestra el mensaje \getElementById[MSG]{MSG5}.}
			\Titem \UCerr{Tres}{El cliente introduce mal la longitud de un campo}{Se muestra el mensaje \getElementById[MSG]{MSG8}.}
		\end{Titemize}
		
	}
	\UCitem{Viene de}{\getElementById[CU]{CUMP2.1}}
	\end{UseCase}
	
	
	\begin{UCtrayectoria}
		\UCpaso[\UCactor] Presiona sobre el texto Agregar forma de pago como se muestra en la pantalla \getElementById[IU]{IUMP2.1.1A}. 
		\UCpaso Muestra la pantalla \getElementById[IU]{IUMP2.1.1B}. 
		\UCpaso[\UCactor] Presiona sobre la opcion PayPal\refTray{A}
		\UCpaso Muestra la pantalla  \getElementById[IU]{IUMP2.1.1C}. 
		\UCpaso[\UCactor]Introduce el \getElementById[Entidad]{paypal.correoElectronico}. de su cuenta de Paypal.
		\UCpaso[\UCactor]Introduce la \getElementById[Entidad]{paypal.contraseña}. de su cuenta de Paypal.
		\UCpaso[\UCactor]Presiona el botón Iniciar sesión.
		\UCpaso Muestra la pantalla \getElementById[IU]{IUMP2.1.1A}. con las formas de pago previamente agregadas junto con la que se acaba de registrar.
	\end{UCtrayectoria}
	

	\begin{UCtrayectoriaA}{A}{Selección de forma de pago tarjeta de crédito o débito}
		\UCpaso[\UCactor]Presiona sobre la opción Tarjeta de crédito o débito .
		\UCpaso Muestra la pantalla  \getElementById[IU]{IUMP2.1.1D}.
		 \UCpaso[\UCactor]Introduce el \getElementById[Entidad]{tarjeta.numeroTarjeta}.
		 \UCpaso[\UCactor]Introduce la \getElementById[Entidad]{tarjeta.fechaVencimiento}.
		 \UCpaso[\UCactor]Introduce la \getElementById[Entidad]{tarjeta.claveSeguridad}.
		\UCpaso Verifica que los datos son correctos. \refTray{B}\refTray{C}\refTray{D}
		\UCpaso Regresa al paso 8 de la \refTray{Principal}
	\end{UCtrayectoriaA}

	\begin{UCtrayectoriaA}{B}{EL formato de los datos introducidos por el cliente no son correctos}
		\UCpaso Se muestra el mensaje asociado al error \refErr{Uno}
		\UCpaso Regresa al paso 2 de la \refTray{A}.
	\end{UCtrayectoriaA}
	
\begin{UCtrayectoriaA}{C}{No se introdujeron algun o algunos campos obligatorios}
		\UCpaso Se muestra el mensaje asociado al error \refErr{Dos}
		\UCpaso Regresa al paso 2 de la \refTray{A}.
	\end{UCtrayectoriaA}

\begin{UCtrayectoriaA}{D}{La longitud de un campo sobrepasa o carece del tamaño requerido}
		\UCpaso Se muestra el mensaje asociado al error \refErr{Tres}
		\UCpaso Regresa al paso 2 de la \refTray{A}.
	\end{UCtrayectoriaA}


	