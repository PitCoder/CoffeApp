%!TEX root = ../../prueba.tex
\begin{IU}{MN01}{Menú del Cliente}{Este menú tiene como principal objetivo mostrar las opciones a las que el \getElementById[Actor]{Cliente} tiene acceso en la aplicación móvil. Las opciones que este menú contiene son las más relevantes y por consecuencia las de fácil acceso para el cliente.}{menu/mn_cliente/MN01}
	\item[Acciones:]\hspace{1pt}
	\begin{Citemize}
	%	\item \homeIcon{} Al presionar este icono se muestran los locales más cercanos a la ubicación del cliente o también permite realizar la búsqueda de un local por nombre, ambas interacciones están descritas en el caso de uso \getElementById[CU]{CUMC3} y en la pantalla \getElementById[IU]{IUMC3}.
		\item \ordersIcon{} Al presionar este icono se muestran las ordenes que el cliente ha hecho en la aplicación así como también consultar la orden cuyo estado actual es activo, ambas interacciones están descritas en el caso de uso \getElementById[CU]{CUMC2.3} y en la pantalla \getElementById[IU]{IUMC2.3}.
		\item \cartIcon{} Al presionar este icono se muestran los productos que han sido seleccionados para el consumo del cliente así como con el fin de permitirle quitar o finalizar una compra y que esta llegue a la cocina para ser preparada, ambas interacciones están descritas en el \getElementById[CU]{CUMC4} y en la pantalla \getElementById[IU]{IUMC4}.
		\item \userAccountIcon{} Al presionar este icono se le desplegará otro menú al cliente en el que podrá administrar la información de su cuenta como cambiar su contraseña o su foto de perfil, administrar sus formas de pago o cerrar sesión en la aplicación.
	\end{Citemize}


\end{IU}